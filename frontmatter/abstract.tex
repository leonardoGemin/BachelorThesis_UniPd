%!TEX root = ../dissertation.tex
% the abstract

\newthought{Con l'evolversi della tecnologia,} oggetti sempre più piccoli necessitano soluzioni in grado di comunicare dati sensibili senza il timore che tali dati possano essere letti da persone non autorizzate.

Proprio per questo motivo, si è resa necessaria una reimplementazione del crittosistema RSA, letto in chiave \emph{embedded}, ovvero in condizioni di scarsa memoria disponibile e in oggetti che hanno esclusivamente una o poche funzionalità.

Il crittosistema, sviluppato in linguaggio C, non fa uso di librerie esterne oltre a quelle standard. Questo è importante perché permette di avere l'esatta computazione del codice, in modo da poterne frammentare l'esecuzione nel caso in cui venisse eseguito su un dispositivo embedded.

Le varie funzioni sono state progettate per ottenere un giusto compromesso tra la complessità computazionale e l'utilizzo di memoria, per cui alcune procedure, pur presentando complessità non ottimali, si rivelano estremamente adatte al problema esposto. È il caso, ad esempio, della moltiplicazione, presentata nei capitoli sucessivi, caratterizzata da una complessità quadratica, molto più degli algoritmi più recenti, a partire da quello di Karatsuba, passando per la procedura di Toom-Cook e per l'algoritmo di Sch\"onhage-Strassen, fino alla trasformata di F\"urer. 

In relazione alla complessità computazionale è il tempo. Dopo numerosi tentativi, manipolazioni e ristrutturazioni del codice, si è riusciti a far svolgere al calcolatore le operazioni per la generazione delle chiavi cifrate in tempi pressoché ragionevoli, come illustrato nei capitoli a seguire.

