%!TEX root = ../dissertation.tex
\chapter{Conclusione}

Tra la prime difficoltà incontrate nello sviluppo del crittosistema, spicca l'implementazione degli algoritmi di operazioni matematiche tra numeri che, nei capitoli precedenti, abbiamo visto essere composti da \emph{array di interi}. Ciò vuol dire che le operazioni canoniche a disposizione non erano utilizzabili senza adattamenti perfezionati \emph{ad hoc}. I fattori che hanno condizionato la scelta degli algoritmi utilizzati non sono stati esclusivamente di natura prestazionale, ma un giusto compromesso tra complessità computazionale e utilizzo dello spazio a disposizione. Non bisogna dimenticare che questo specifico crittosistema è pensato e sviluppato per un funzionamento su \emph{sistemi embedded}, quindi in condizioni di memoria RAM molto limitata.

Una volta definiti gli algoritmi che permettessero l'interazione matematica tra le variabili, e dopo aver cercato di ridurre al minimo le operazioni e le variabili temporanee, si è passati a definire tutti quegli algoritmi che servono al funzionamento, in senso stretto, del crittosistema, quindi algoritmi per la generazione di numeri casuali, per il controllo di primalità, per la generazione finale delle chiavi pubbliche e private. Anche per tali algoritmi si è cercato un compromesso che rendesse accettabile il tempo di attesa, senza inficiare troppo sullo spazio di memoria.

Tra le tante difficoltà, il raggiungimento dei vari obiettivi prefissati è stato di aiuto per risolvere gli ulteriori errori e \emph{bug} che, bene o male, sono spesso presenti nel cammino che porta da un'idea allo sviluppo del prodotto. Uno scoglio molto importante è stato lo sviluppo dell'algoritmo adibito al test di primalità dei numeri casuali: da una complessità computazionale che oscillava tra il polinomiale e l'esponenziale, si è riusciti, attraverso a continue modifiche nell'algoritmo, a portare l'esecuzione a tempi di attesa accettabili. 

Senza ombra di dubbio sono diverse le modifiche ed i miglioramenti che possono essere attuati al crittosistema sviluppato, ad esempio alcuni algoritmi possono essere riscritti in modo tale da essere ancor più performanti, oppure è possibile convertire in iterativi i pochi algoritmi ricorsivi presenti, preferibili in quanto creano meno variabili con un conseguente risparmio di memoria.

Complessivamente, non è stato un lavoro banale, ma nemmeno troppo difficile. La comprensione degli argomenti trattati, il capire i meccanismi, nascosti agli utenti finali, che permettono il funzionamento del sistema ha fatto sì che si instaurasse una curiosità molto profonda del sistema stesso. Motivo per cui il lavoro può considerarsi concluso se fine a se stesso, ma è da pensare come l'inizio di un nuovo cammino, se si pensa a tutte le applicazioni che fanno da corollario e che completano il crittosistema RSA.